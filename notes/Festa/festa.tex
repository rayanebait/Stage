\documentclass[12pt]{article}
\usepackage{amsmath, latexsym, amsfonts,amssymb, amsthm,
    amscd, geometry, xspace, enumerate,mathtools}

\usepackage[dvipsnames]{xcolor}
\usepackage{quiver, tikz}
\usepackage{mathrsfs}
\usepackage{hyperref}

\definecolor{wgrey}{RGB}{148, 38, 55}

\setlength{\oddsidemargin}{-10mm}
\setlength{\evensidemargin}{5mm}
\setlength{\textwidth}{175mm}
\setlength{\headsep}{0mm}
\setlength{\topmargin}{0mm}
\setlength{\textheight}{220mm}
\setlength\parindent{24pt}

\newcommand{\Z}{\mathbb{Z}}
\newcommand{\R}{\mathbb{R}}
\newcommand{\rel}{\omathcal{R}}
\newcommand{\Q}{\mathbb{Q}}
\newcommand{\C}{\mathbb{C}}
\newcommand{\N}{\mathbb{N}}
\newcommand{\K}{\mathbb{K}}
\newcommand{\A}{\mathbb{A}}
\newcommand{\B}{\mathcal{B}}
\newcommand{\stSheaf}{\mathcal{O}}

\newcommand{\cL}{\mathscr{L}}
\newcommand{\F}{\mathscr{F}}
\newcommand{\G}{\mathscr{G}}
\newcommand{\D}{\mathscr{D}}
\newcommand{\E}{\mathscr{E}}

\theoremstyle{plain}
\newtheorem{thm}[subsubsection]{Theorem}
\newtheorem{lem}[subsubsection]{Lemma}
\newtheorem{prop}[subsubsection]{Proposition}
\newtheorem{cor}[subsubsection]{Corollary}

\theoremstyle{definition}
\newtheorem{defn}[subsubsection]{Definition}
\newtheorem{rmq}[subsubsection]{Remark}
\newtheorem{conj}[subsubsection]{Conjecture}
\newtheorem{exmp}[subsubsection]{Examples}
\newtheorem{quest}[subsubsection]{Exercises}

\theoremstyle{remark}
\newtheorem{rem}{Remark}
\newtheorem{note}{Note}

\hypersetup{
    colorlinks=true,
    linkcolor=blue,
    urlcolor=wgrey,
    filecolor=wgrey
}

\definecolor{wgrey}{RGB}{148, 38, 55}


\begin{document}

\section{The trapdoor}
It works as follows, $A,B\in \mathcal M_b$ will be scalar matrices.\\

\noindent\textbf{Public parameters}: $E_0, d_1, d_2$.\\
\noindent\textbf{Trapdoor parameters (public key, secret key)}:
$(pk, sk)=(E_A, (A,\phi_A))$.\\
\noindent \textbf{Input of the trapdoor:} $K_1\subset E_0[d_1], K_2\subset E_A[d_2]$, $B$.\\

\noindent \textbf{Evaluation:}
% https://q.uiver.app/#q=WzAsMTIsWzQsMSwiRV9BIl0sWzQsMywiRV8yIl0sWzUsMywiQipBKlxcYmVnaW57cG1hdHJpeH1cXHBoaV8yXFxjaXJjXFxwaGlfQShQX2IpXFxcXCBcXHBoaV8yXFxjaXJjXFxwaGlfQShRX2IpXFxlbmR7cG1hdHJpeH0iXSxbNSwxLCJBKlxcYmVnaW57cG1hdHJpeH1cXHBoaV9BKFBfYilcXFxcIFxccGhpX0EoUV9iKVxcZW5ke3BtYXRyaXh9Il0sWzIsMSwiRV8wIl0sWzIsMywiRV8xIl0sWzEsMywiQipcXGJlZ2lue3BtYXRyaXh9XFxwaGlfMShQX2IpXFxcXCBcXHBoaV8xKFFfYilcXGVuZHtwbWF0cml4fSJdLFsxLDEsIlxcYmVnaW57cG1hdHJpeH1QX2JcXFxcIFFfYlxcZW5ke3BtYXRyaXh9Il0sWzAsMywiXFxiZWdpbntwbWF0cml4fVJfMVxcXFwgU18xXFxlbmR7cG1hdHJpeH0iXSxbNiwzLCJcXGJlZ2lue3BtYXRyaXh9Ul8yXFxcXCBTXzJcXGVuZHtwbWF0cml4fSJdLFsyLDAsIktfMSJdLFs0LDAsIktfMiJdLFswLDEsIlxccGhpXzIiLDJdLFsyLDEsIlxcbmkiLDEseyJzdHlsZSI6eyJib2R5Ijp7Im5hbWUiOiJub25lIn0sImhlYWQiOnsibmFtZSI6Im5vbmUifX19XSxbMywwLCJcXG5pIiwxLHsic3R5bGUiOnsiYm9keSI6eyJuYW1lIjoibm9uZSJ9LCJoZWFkIjp7Im5hbWUiOiJub25lIn19fV0sWzQsNSwiXFxwaGlfMSJdLFs2LDUsIlxcaW4iLDEseyJzdHlsZSI6eyJib2R5Ijp7Im5hbWUiOiJub25lIn0sImhlYWQiOnsibmFtZSI6Im5vbmUifX19XSxbNyw0LCJcXGluIiwxLHsic3R5bGUiOnsiYm9keSI6eyJuYW1lIjoibm9uZSJ9LCJoZWFkIjp7Im5hbWUiOiJub25lIn19fV0sWzgsNiwiPSIsMSx7InN0eWxlIjp7ImJvZHkiOnsibmFtZSI6Im5vbmUifSwiaGVhZCI6eyJuYW1lIjoibm9uZSJ9fX1dLFs5LDIsIj0iLDEseyJzdHlsZSI6eyJib2R5Ijp7Im5hbWUiOiJub25lIn0sImhlYWQiOnsibmFtZSI6Im5vbmUifX19XSxbMTAsNCwiXFxiaWdjYXAiLDEseyJzdHlsZSI6eyJib2R5Ijp7Im5hbWUiOiJub25lIn0sImhlYWQiOnsibmFtZSI6Im5vbmUifX19XSxbMTEsMCwiXFxiaWdjYXAiLDEseyJzdHlsZSI6eyJib2R5Ijp7Im5hbWUiOiJub25lIn0sImhlYWQiOnsibmFtZSI6Im5vbmUifX19XV0=
\[\begin{tikzcd}
	&& {K_1} && {K_2} \\
	& \begin{array}{c} \begin{pmatrix}P_b\\ Q_b\end{pmatrix} \end{array} & {E_0} && {E_A} & \begin{array}{c} A*\begin{pmatrix}\phi_A(P_b)\\ \phi_A(Q_b)\end{pmatrix} \end{array} \\
	\\
	\begin{array}{c} \begin{pmatrix}R_1\\ S_1\end{pmatrix} \end{array} & \begin{array}{c} B*\begin{pmatrix}\phi_1(P_b)\\ \phi_1(Q_b)\end{pmatrix} \end{array} & {E_1} && {E_2} & \begin{array}{c} B*A*\begin{pmatrix}\phi_2\circ\phi_A(P_b)\\ \phi_2\circ\phi_A(Q_b)\end{pmatrix} \end{array} & \begin{array}{c} \begin{pmatrix}R_2\\ S_2\end{pmatrix} \end{array}
	\arrow["\bigcap"{description}, draw=none, from=1-3, to=2-3]
	\arrow["\bigcap"{description}, draw=none, from=1-5, to=2-5]
	\arrow["\in"{description}, draw=none, from=2-2, to=2-3]
	\arrow["{\phi_1}", from=2-3, to=4-3]
	\arrow["{\phi_2}"', from=2-5, to=4-5]
	\arrow["\ni"{description}, draw=none, from=2-6, to=2-5]
	\arrow["{=}"{description}, draw=none, from=4-1, to=4-2]
	\arrow["\in"{description}, draw=none, from=4-2, to=4-3]
	\arrow["\ni"{description}, draw=none, from=4-6, to=4-5]
	\arrow["{=}"{description}, draw=none, from=4-7, to=4-6]
\end{tikzcd}\]

\noindent \textbf{Output:} $(E_1, R_1, S_1, E_2, R_2, S_2)$.\\

\noindent Now to inverse the map, we have access to:
\begin{itemize}
    \item the secret key $(A, \phi_A: E_0\to E_A)$
    \item the image points \[\begin{pmatrix}R_1\\ S_1\end{pmatrix}=B*\begin{pmatrix}\phi_1(P_b)\\ \phi_1(Q_b)\end{pmatrix}\]
    \item and the image points \[\begin{pmatrix}R_2\\ S_2\end{pmatrix}=B*A*\begin{pmatrix}\phi_2\circ\phi_A(P_b)\\ \phi_2\circ\phi_A(Q_b)\end{pmatrix}\]
\end{itemize}
Consider $\psi=\phi_2\circ\phi_A\circ \widehat{\phi_1}$, which 
has degree $=d_2d_Ad_1$, we have 
\[
    B*A*B^{-1}*\begin{pmatrix}\psi(R_1)\\ \psi(S_1)\end{pmatrix}=[d_1]\begin{pmatrix}R_2\\ S_2\end{pmatrix}
\]
But $A$ and $B$ commute so that since we know $A$ and $d_1$ we can 
recover $\begin{pmatrix}\psi(R_1)\\ \psi(S_1)\end{pmatrix}$.
Now we would like to recover $\psi$. We have the torsion point images
of order $2^b>d_1d_Ad_2$ so that we can apply the usual 
torsion attacks. The parameters just need to be worked out.\\
\newpage
It happens that, under the CIST² assumption, 
the FESTA trapdoor verifies the following definition:
\begin{defn}[Quantum partial domain one-way function]
    Let $X_0$, $X_1$ and Y be three finite sets. A function $f : X_0\times X_1\to Y$
is a quantum partial-domain one-way function if, for any polynomial-time quantum adversary $A$,
the following holds:
\[P(s'=s|s\leftarrow X_0,t\leftarrow X_1, s'\leftarrow A(f(s,t)))\]
\end{defn}

Then the OAEP transform, as described \href{https://eprint.iacr.org/2021/237.pdf}{here}
builds a PKE from such a trapdoor functions.
\section{The concrete instantiation}
Consider the diagram: 
% https://q.uiver.app/#q=WzAsNSxbMCwwLCJFXzAiXSxbMSwwLCJcXHRpbGRle0V9X0EiXSxbMiwwLCJFX0EiXSxbMCwyLCJFXzEiXSxbMiwyLCJFXzIiXSxbMSwwLCJcXHdpZGVoYXR7XFxwaGl9X3tBLDF9Il0sWzEsMiwiXFxwaGlfe0EsMn0iLDJdLFswLDMsIlxccGhpXzEiLDJdLFsyLDQsIlxccGhpXzIiXSxbMywzLCJbbV8xXSIsMix7InJhZGl1cyI6LTN9XSxbNCw0LCJbbV8yXSIsMix7InJhZGl1cyI6LTN9XSxbMCwyLCJcXHBoaV9BIiwwLHsiY3VydmUiOi0zfV1d
\[\begin{tikzcd}
	{E_0} & {\tilde{E}_A} & {E_A} \\
	\\
	{E_1} && {E_2}
	\arrow["{\phi_A}", curve={height=-18pt}, from=1-1, to=1-3]
	\arrow["{\phi_1}"', from=1-1, to=3-1]
	\arrow["{\widehat{\phi}_{A,1}}", from=1-2, to=1-1]
	\arrow["{\phi_{A,2}}"', from=1-2, to=1-3]
	\arrow["{\phi_2}", from=1-3, to=3-3]
	\arrow["{[m_1]}"', from=3-1, to=3-1, loop, in=305, out=235, distance=10mm]
	\arrow["{[m_2]}"', from=3-3, to=3-3, loop, in=305, out=235, distance=10mm]
\end{tikzcd}\]
Where we decomposed $\phi_A$ in $\phi_{A,1}\circ\phi_{A,2}$. The main
idea here would be to directly get 

\[\phi_2\circ\phi_A\circ \widehat{\phi}_1\]

from the the $2^b$ torsion point images. We would need to have $2^b-d_2*
d_A*d_1$ smooth. Which gives few choices and usually low efficiency.
Instead, since we already have $\phi_A$ to invert the trapdoor. The idea
would be to decompose $\phi_A$ as $\phi_{A,1}\circ\phi_{A,2}$ and use the
hidden diagram

% https://q.uiver.app/#q=WzAsNixbMCwwLCJFXzAiXSxbNCwwLCJFX0EiXSxbMCw0LCJFXzEiXSxbNCw0LCJFXzIiXSxbMiw1LCJFX3sxMn0iXSxbMiwxLCJcXHRpbGRle0V9X0EiXSxbMCw1LCJcXHBoaV97QSwxfSJdLFs1LDEsIlxccGhpX3tBLDJ9Il0sWzAsMiwiXFxwaGlfMSIsMl0sWzIsNCwiZ197ZF97QSwyfWRfMn0iLDJdLFs0LDMsImdfe2Rfe0EsMX1kXzF9IiwyXSxbMSwzLCJcXHBoaV8yIl0sWzUsMiwiXFxwaGlfMVxcY2lyY1xcd2lkZWhhdHtcXHBoaX1fe0EsMX0iLDFdLFs1LDMsIlxccGhpXzJcXGNpcmNcXHBoaV97QSwyfSIsMV1d
\[\begin{tikzcd}
	{E_0} &&&& {E_A} \\
	&& {\tilde{E}_A} \\
	\\
	\\
	{E_1} &&&& {E_2} \\
	&& {E_{12}}
	\arrow["{\phi_{A,1}}", from=1-1, to=2-3]
	\arrow["{\phi_1}"', from=1-1, to=5-1]
	\arrow["{\phi_2}", from=1-5, to=5-5]
	\arrow["{\phi_{A,2}}", from=2-3, to=1-5]
	\arrow["{\phi_1\circ\widehat{\phi}_{A,1}}"{description}, from=2-3, to=5-1]
	\arrow["{\phi_2\circ\phi_{A,2}}"{description}, from=2-3, to=5-5]
	\arrow["{g_{d_{A,2}d_2}}"', from=5-1, to=6-3]
	\arrow["{g_{d_{A,1}d_1}}"', from=6-3, to=5-5]
\end{tikzcd}\]

In which finding good parameters amounts to solving 
\[m_1^2d_{A,1}d_1+m_2^2d_{A,2}d_2=2^b\]
The trick of decomposing with some scalar multiplication doesn't change
anything to the security! The paper proposes a way to find solutions 
with the desired properties efficiently. The $d_i's$ are all squares.


\end{document}
