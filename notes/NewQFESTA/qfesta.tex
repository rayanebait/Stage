\documentclass[12pt]{article}
\usepackage{amsmath, latexsym, amsfonts,amssymb, amsthm,
    amscd, geometry, xspace, enumerate,mathtools}

\usepackage[dvipsnames]{xcolor}
\usepackage{quiver, tikz}
\usepackage{mathrsfs}
\usepackage{hyperref}

\definecolor{wgrey}{RGB}{148, 38, 55}

\setlength{\oddsidemargin}{-10mm}
\setlength{\evensidemargin}{5mm}
\setlength{\textwidth}{175mm}
\setlength{\headsep}{0mm}
\setlength{\topmargin}{0mm}
\setlength{\textheight}{220mm}
\setlength\parindent{24pt}

\newcommand{\Z}{\mathbb{Z}}
\newcommand{\R}{\mathbb{R}}
\newcommand{\rel}{\omathcal{R}}
\newcommand{\Q}{\mathbb{Q}}
\newcommand{\C}{\mathbb{C}}
\newcommand{\N}{\mathbb{N}}
\newcommand{\K}{\mathbb{K}}
\newcommand{\A}{\mathbb{A}}
\newcommand{\B}{\mathcal{B}}
\newcommand{\stSheaf}{\mathcal{O}}

\newcommand{\cL}{\mathscr{L}}
\newcommand{\F}{\mathscr{F}}
\newcommand{\G}{\mathscr{G}}
\newcommand{\D}{\mathscr{D}}
\newcommand{\E}{\mathscr{E}}

\theoremstyle{plain}
\newtheorem{thm}[subsubsection]{Theorem}
\newtheorem{lem}[subsubsection]{Lemma}
\newtheorem{prop}[subsubsection]{Proposition}
\newtheorem{cor}[subsubsection]{Corollary}

\theoremstyle{definition}
\newtheorem{defn}[subsubsection]{Definition}
\newtheorem{rmq}[subsubsection]{Remark}
\newtheorem{conj}[subsubsection]{Conjecture}
\newtheorem{exmp}[subsubsection]{Examples}
\newtheorem{quest}[subsubsection]{Exercises}

\theoremstyle{remark}
\newtheorem{rem}{Remark}
\newtheorem{note}{Note}

\hypersetup{
    colorlinks=true,
    linkcolor=blue,
    urlcolor=wgrey,
    filecolor=wgrey
}

\definecolor{wgrey}{RGB}{148, 38, 55}


\begin{document}
\section{the new scheme}
Le basique c'est celui là où on prend des isogénies radicales des
deux côtés.
% https://q.uiver.app/#q=WzAsNCxbMCwwLCJFX3MiXSxbMywwLCJcXHRpbGRle0V9X3MiXSxbMCwyLCJFXzEiXSxbMywyLCJFXzIiXSxbMCwxLCJkX0EiXSxbMCwyLCIzXmIiLDJdLFsxLDMsIjNeYiJdXQ==
\[\begin{tikzcd}
	{E_s} &&& {\tilde{E}_s} \\
	\\
	{E_1} &&& {E_2}
	\arrow["{d_A}", from=1-1, to=1-4]
	\arrow["{3^2b}"', from=1-1, to=3-1]
	\arrow["{3^2b}", from=1-4, to=3-4]
\end{tikzcd}\]

On cherche ducoup $a$ et $b$ t.q. $2^{6a}-3^{4b}d_A=x^2+y^2$. Pour ça 
on fait varier $d_A$ de $3^{2b}-log(p)$ à $3^{2b}-log(p)$ et on teste 
si c'est premier (bcp plus de chance de trouver un premier qu'un composé
avec la bonne tête, et surtout c'est prouvable). Ducoup 
\[p=2^{6a}3^{2b}f-1\]
pour un \textbf{chiffrement ultra rapide} mais un déchiffrement super 
long. Ou
\[p=2^{6a}3f-1\]
pour un chiffrement plus lent et un déchiffrement plus rapide.

\end{document}

